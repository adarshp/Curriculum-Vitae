% =============================================================================
% Curriculum Vitae - Adarsh Pyarelal
% Based on Kieran Healy's CV
% =============================================================================
\documentclass[final,oneside,10pt]{memoir}

% Silence some noisy warnings temporarily
\usepackage{silence}
\WarningFilter{fontaxes}{I don't know how to decode}
\WarningFilter{latex}{Marginpar}
\WarningFilter{hyperref}{subfigure}
\usepackage[tracking]{microtype}
\usepackage[smaller]{acronym}
\renewcommand*{\aclabelfont}[1]{\acsfont{\scshape #1}}
\usepackage{pgfornament}

% =============================================================================
% If using Minion Pro
% =============================================================================

\usepackage[onlymath, minionint, swash,lf]{MinionPro}
\usepackage{fontawesome}
\usepackage{fontspec}
\setmainfont[%
  Ligatures = {Common, TeX},Numbers = {OldStyle, Proportional},
  %ItalicFeatures={Style=Swash},
  SmallCapsFeatures={Letters=SmallCaps,LetterSpace=6},
]{Adobe Jenson Pro}
\setsansfont[%
Numbers = {OldStyle, Proportional},
BoldFont={ScalaSans Bold},
]{Scala Sans}
\setmonofont{Menlo}[Scale=MatchLowercase]
\usepackage{pifont} % To get Minion Pro ornaments

% =============================================================================
% If not using Minion Pro, uncomment the lines below, and comment out the lines
% in the block above.
% =============================================================================
% \usepackage[T1]{fontenc}
% \usepackage{mathpazo}
% \newcommand{\slantfrac}[2]{\frac{#1}{#2}}
% =============================================================================

\usepackage[toc]{tabfigures}
% \PassOptionsToPackage{utf8}{inputenc}
	% \usepackage{inputenc}

\PassOptionsToPackage{fleqn}{amsmath}   % math environments and more by the AMS 
    \usepackage{amsmath}

% =============================================================================
% Colors
% =============================================================================

\PassOptionsToPackage{dvipsnames}{xcolor}
    \RequirePackage{xcolor} % [dvipsnames] 
\definecolor{webgreen}{rgb}{0,.5,0}
\definecolor{webbrown}{rgb}{.6,0,0}
\definecolor{Maroon}{cmyk}{0, 0.87, 0.68, 0.32}
\definecolor{RoyalBlue}{cmyk}{1, 0.50, 0, 0}
\definecolor{Black}{cmyk}{0, 0, 0, 0}
\definecolor{shadecolor}{gray}{0.9}

% =============================================================================
% Set up hyperlinks
% =============================================================================

\PassOptionsToPackage{pdftex,hyperfootnotes=false,pdfpagelabels=true}{hyperref}
    \usepackage{hyperref}  % backref linktocpage pagebackref
\pdfcompresslevel=9
\pdfadjustspacing=1 

\hypersetup{%
    colorlinks=true, linktocpage=true, pdfstartpage=3, pdfstartview=FitV,%
    breaklinks=true, pdfpagemode=UseNone, pageanchor=true, pdfpagemode=UseOutlines,%
    plainpages=false, bookmarksnumbered, bookmarksopen=true, bookmarksopenlevel=1,%
    hypertexnames=true, pdfhighlight=/O,%nesting=true,%frenchlinks,%
    urlcolor=RoyalBlue, linkcolor=RoyalBlue, citecolor=webgreen, %pagecolor=RoyalBlue,%
    pdftitle={Curriculum Vitae - Adarsh Pyarelal},%
    pdfauthor={Adarsh Pyarelal},%
    pdfsubject={},%
    pdfkeywords={},%
    pdfcreator={pdfLaTeX},%
    pdfproducer={LaTeX}%
}   

% =============================================================================
% Metadata
% =============================================================================

\def\myauthor{Adarsh Pyarelal}
\def\mytitle{Curriculum Vitae}
\def\mycopyright{\myauthor}
\def\mykeywords{}
\def\mybibliostyle{plain}
\def\mybibliocommand{}
\def\mysubtitle{}
\def\myaffiliation{University of Arizona}
\def\myaddress{Department of Physics}
\def\myemail{adarsh@email.arizona.edu}
\def\myweb{www.physics.arizona.edu/$\sim$adarsh}
\date{} % not used (revision control instead)

% =============================================================================
% Memoir package - layout and styling
% =============================================================================

% Here we set typeblock widths for the main body and the footnotes
\setlxvchars[\normalfont] % about 66 characters per column
\setxlvchars[\normalfont\footnotesize] % about 45 characters per column

% Set outer and spine margins (designed for Minion Pro 10pt, change accordingly
% for different fonts. A wide margin is chosen both for legibility of the
% typeblock and for tight integration of marginfigures and margin footnotes.
\setlrmarginsandblock{1.5in}{1.5in}{} % This sets \textwidth to 281.0 pt

% Set upper and lower margins
\setulmarginsandblock{1.22in}{1.22in}{*}

% Set properties of margin notes, sidecaptioned floats, and footnotes in the
% margin.
\setmarginnotes{0.2in}{1.9in}{2\onelineskip}
\setsidecaps{0.2in}{1.9in}
\sidecapmargin{outer}
\renewcommand*{\sidecapstyle}{\normalfont\footnotesize}
\setsidecappos{c}

% Set footnotes in the margin rather than at the foot of the page
\footnotesinmargin
\setsidefeet{\marginparsep}{1.9in}{0.2in}{0pt}{\flushleftright\footnotesize}{*}

% Integrate the counters of the sidefootnotes and footnotes in margin.
\letcountercounter{sidefootnote}{footnote}
\setlength{\footmarkwidth}{0em}
\setlength{\footmarksep}{-\footmarkwidth}
\setlength{\footparindent}{1em}
\sideparmargin{outer}

\renewcommand*{\sideparfont}{\color{Maroon}\itshape}
\renewcommand*{\sideparvshift}{2\baselineskip}
\marginparmargin{outer}

% Style the entries in the Table of Contents
\renewcommand*{\contentsname}{Table of Contents}
\renewcommand*{\cftchapterfont}{\scshape\MakeTextLowercase}
\renewcommand*{\cftpartfont}{\color{Maroon}\scshape\MakeTextLowercase}
\captionstyle[\centering]{\sffamily\footnotesize}
\captionnamefont{\color{Maroon}\footnotesize\sffamily}
% Reduce spacing between list items
\tightlists

% Make marginfigures centered by default
\setfloatadjustment{marginfigure}{\centering}

% Headers and footers - page numbers, section headings, etc.
\makepagestyle{tufte}
\createmark{chapter}{left}{nonumber}{}{}
\createmark{section}{right}{nonumber}{}{}
\makeoddhead{tufte}{}{}{\scshape\MakeTextLowercase{\leftmark}~~|~~\thepage}
\makeevenhead{tufte}{\thepage~~|~~\scshape\MakeTextLowercase{\rightmark}}{}{}
 \makerunningwidth{tufte}[8in]{8in}
%\aliaspagestyle{chapter}{empty}
\nouppercaseheads
\pagestyle{tufte}

\setsecheadstyle{\Large\scshape\hspace{0.54in}\MakeTextLowercase}
% Styling Part headings
\renewcommand{\parttitlefont}{\color{Maroon}\normalfont\scshape%
                              \MakeTextLowercase}
\renewcommand{\partnamefont}{\normalfont}
\renewcommand{\partnumfont}{\normalfont}
\setafterparaskip{-0.5em}

% Setting up figures to allow subfloats
\newsubfloat{figure}
\checkandfixthelayout

% =============================================================================
% Misc layout commands
% =============================================================================

\newcolumntype{R}[1]{>{\raggedleft}p{#1}}
\linespread{1.1}

% =============================================================================

\begin{document}
%%%------------------------------------------------------------------------
%%% Address and contact block
%%%------------------------------------------------------------------------
\pagestyle{empty}
{\noindent \LARGE\scshape\color{Maroon}{\MakeTextLowercase\myauthor}}
\bigskip
{\color{gray}\hrule}
\bigskip
{\small
\noindent \myaddress \hfill  \texttt{\href{mailto:\myemail}{\myemail}} \, \faEnvelope~\\
\myaffiliation \hfill \texttt{\href{http://www.physics.arizona.edu/\~adarsh}{\myweb}} \, \faGlobe~\\
Tucson, \textsc{az} 85719  \hfill \texttt{\href{http://www.github.com/adarshp}{http://www.github.com/adarshp}} ~\faGithub~\\
}
{\color{gray}\hrule}
\medskip

\section*{Research Interests}
\begin{tabularx}{\linewidth}{R{0.5in}X}
& Theoretical particle physics, machine learning, computer vision, Bayesian modeling.
\end{tabularx}
\section*{Education}
\newcommand{\degree}[6]{
  \emph{#1} & {\bfseries \sffamily #2}\\
  & {\sffamily #3}\\
  & Advisor: {\scshape \MakeTextLowercase{#4}}\\
  & Thesis: \href{#5}{\emph{#6}}\\
}
\begin{tabularx}{\linewidth}{R{0.5in}X}
\degree{2011-17}%
  {University of Arizona}%
  {Ph. D. in Physics}%
  {Shufang Su}%
  {https://github.com/adarshp/dissertation}%
  {Hidden Higgses and Dark Matter at Current and Future Colliders}\\
\degree{2007-11}%
  {Reed College}%
  {B. A. in Physics}%
  {Nelia Mann}%
  {http://www.physics.arizona.edu/~adarsh/research/reedthesis/}%
  {Contribution of the neutral pion Regge trajectory to the exclusive central production of $\eta$(548) mesons in high energy proton/proton collisions}
\end{tabularx}
\section*{Honors and Awards}
\newcommand{\award}[2]{\emph{#1} & #2\\}
\begin{tabularx}{\linewidth}{R{0.5in}X}
\award{2016,17}{Physics Publications/Presentations Award}
\award{2016}{Galileo Circle Scholarship}
\award{2015}{Graduate and Professional Student Council Travel Award}
\award{}{Professor C. Y. Fan `FanFare' Travel Award}
\award{}{Graduate College Fellowship in Physics}
\award{2014-16}{APS 4CS Student Travel Grant}
\award{2014}{Outstanding Graduate Student Colloquium Presentation}
\end{tabularx}
\section*{Publications}
\newcommand{\publication}[4]{
  \emph{#1} & {\itshape #2}\\
  & #3\\ % Author(s)
  & #4 % Journal
}
\begin{tabularx}{\linewidth}{R{0.5in}X}
\publication{2017}{A Razor Search for Bino Dark Matter at 100 TeV}%
    {A. Pyarelal, and S. Su (in preparation)}%
    {}\\
\publication{}{Exotic Higgs Decays at 14 and 100 TeV}%
    {F. Kling, H. Li, A. Pyarelal, H. Song, and S. Su (in preparation)}%
    {}\\
\publication{2015}{Light Charged Higgs Bosons to AW/HW via Top Decay}%
    {F. Kling, A. Pyarelal, and S. Su}%
    {\href{http://link.springer.com/article/10.1007\%2FJHEP11\%282015\%29051}{Journal of High Energy Physics, \textbf{11} (2015) 051}}
\end{tabularx}
\section*{Talks}
\newcommand{\talk}[2]{\emph{#1} & #2\\}
\begin{tabularx}{\linewidth}{R{0.5in}X}
  \talk{2016}{Tucson Data Science Meetup, Tucson, AZ}
  \talk{2015}{Phenomenology 2015 Symposium, University of Pittsburgh}
  \talk{2014-16}{Annual Meeting(s) of the APS Four Corners Section}
  \talk{2014}{23$^{\text{rd}}$ International Conference on Supersymmetry and Unification of Fundamental Interactions, %
              Lake Tahoe, CA}
\end{tabularx}
\section*{Teaching}
\newcommand{\experience}[3]{
  \emph{#1} & #2\\
            & #3\\
}
\begin{tabularx}{\linewidth}{R{0.5in}X}
  \experience{2011-17}{\textsf{\textbf{Teaching Assistant}}, \textsf{ University of Arizona}}
    {
      \begin{itemize}
        \item Physics 105 (Introduction to Scientific Computing), Spring 2017
        \item Physics 381 (Advanced Lab), Fall 2013-present.
             Created various supplemental instructional materials including a \href{http://www.physics.arizona.edu/~adarsh/teaching/phys381}{website} with tips for the experiments,
and a \href{http://www.physics.arizona.edu/~adarsh/teaching/phys381/Error\_Analysis\_Notebook/index.html}{tutorial} on how to perform error analysis and data visualization with Python.
\item  Lecturer for Physics 102 (Introductory Physics for non-majors), Fall 2012.
\item TA for Physics 241/261H Lab: Introductory Electricity and Magnetism, 2011-12.
\item TA for Physics 181, the companion lab to the Physics 102 lecture, Summer 2012 \& Summer 2014.
  \end{itemize}
}
 \end{tabularx}

\section*{Service}
\newcommand{\service}[2]{\emph{#1} & #2\\}
\begin{tabularx}{\linewidth}{R{0.5in}X}
\service{2015}{GPSC Travel Grant Judge }
\service{2012-13}{Member of Physics Grad Council}
\service{}{Member of the Associated Graduate Council for the College of Science}
\service{}{Organized the weekly departmental graduate student seminar series}
\service{}{Arizona Assurance Mentor}
\end{tabularx}
\section*{Skills}
\newcommand{\skill}[2]{\textsc{#1} & #2\\}
\begin{tabularx}{\linewidth}{R{0.5in}X}
\skill{C++} {Wrangling big data with C++.}
  \skill{python}{   Scripting, plotting \& data analysis with \textsc{python} and \texttt{pandas}}
  \skill{git}{ Software version control with \textsc{git} and \href{http://www.github.com/adarshp}{Github}}
\skill{web}{ Website design with \textsc{HTML} and \textsc{CSS}, static site generation.}
\skill{\LaTeX}{ Writing scientific manuscripts and typesetting them with \LaTeX}
\skill{misc.} {Languages: English (native), Hindi, Malayalam}
\end{tabularx}
\section*{Professional Affiliations}
\newcommand{\affiliation}[2]{\textsc{#1} & #2\\}
\begin{tabularx}{\linewidth}{R{0.5in}X}
\affiliation{aps}{American Physical Society}
\end{tabularx}

\end{document}
